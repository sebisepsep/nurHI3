The G faktor is given by:
\begin{align}
    G = 2\sum_i N_i \ln \q{\frac{N_i}{\Tilde{N_i}}}
\end{align}
And the Q Faktor by:
\begin{align}
    Q = 1-P(x,k) = 1 -P(\chi^2 \leq x) = P(\chi^2>x) 
\end{align}

With:
\begin{align}
    k &= N-m \quad \text{binned datapoints - parameter}\\
    P(x,k) &= \frac{\gamma\q{\frac{k}{2},\frac{x}{2}}}{\Gamma\q{\frac{k}{2}}} = \int_0^{\chi^2} p(x,k) =  \q{2^{k/2}\Gamma\f{k}{2}}^{-1} \int_0^{\chi^2} x^{k/2-1}e^{-x/2}
\end{align}

Each dataset has k = 75 - 3, because we are binning data and want to fit a funktion depending on 3 variables to it.  
The G-test compares observed frequency distribution with an expected one, aiming to minimize information loss when approximating true distribution. Comparing Q-values between maximum likelihood and chi-square methods gauges their goodness of fit. A Q-value near 1 indicates both methods offer satisfactory fit. A markedly smaller Q-value for one method suggests its superior fit, better for modeling the data. Significant differences in Q-values indicate disparities in capturing underlying data patterns, warranting further investigation. If Q-values are similar but not near 1 for both methods, alternative models or methods may be needed for better representation.

Unfortunately the previous tasks didn't go as planned, so I couldn't to this task. Another problem was also that some $N_i$ were zero and the logarithm of zero "is" minus inifinity. So this didn't work. A possibility would have been to lower the amounts of bins. This would effect our k value since we have less datapoints in this sense. 
